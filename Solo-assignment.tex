\documentclass[11pt]{article}
\usepackage{lmodern}
\usepackage{microtype}
\usepackage[utf8]{inputenc}
\usepackage[T1]{fontenc}
% Font:
\usepackage{tgpagella,eulervm}

\title{Solo assignment}
\author{Hugh Parsonage}
\date{\today}

\begin{document}

\section{Direction}

\begin{enumerate}
	\item In a three figure group, express the following directions:
	\begin{enumerate}
		\item North: 000
		\item South: 180
		\item East: 090
		\item West: 270
		\item Northeast: 045
		\item Southwest: 315
	\end{enumerate}
	\item Runways are numbered according to their magnetic track. They are abbreviated to a 2 figure group. For example, 
	the southerly runways at Sydney are 16. Because there are parallel runways, they are identified as runways 16 Left and 16 Right.
	\begin{enumerate}
		\item Essendon airport has a runway with a magnetic heading of \(257^\circ\). What would that runway be designated?
		\hfill
		\textbf{26}
		\item Mackay Airport in Queensland has its northern runway magnetic heading as \(319^\circ\). What would this runway be designated as?
		\hfill
		\textbf{32}
		\item Rockhampton's southern runway, heading is \(148^\circ\). What would this runway be designated?
		\hfill
		\textbf{15}
		\item What is the reciprocal of runway 13 at Moorabbin?
		\hfill
		\textbf{31}
	\end{enumerate}
	\item In order to referenfce the position of other traffic or features outside the cockpit we use the clock code. 
	Straight ahead is known as ``12 O'Clock'' and directly behind is our ``6 O'Clock''. Above the horizon is high; 
	behind the horizon is low.
	\begin{enumerate}
		\item If we are told to lookout for aircraft traffic at our ``9 O'Clock'', which window should we look out of?\hfill \textbf{Left}
		\item What about our ``3 O'Clock''.\hfill \textbf{Right}
	\end{enumerate}
	\item Heading is the direction in which the aircraft is pointing, or the angle, measured in degrees clockwise from magnetic north to the 
	longitudinal axis to the aircraft. Simply, the heading can be described as the direction that the aircraft's nose is pointing.

	If North is expressed as 360, what is---
	\begin{enumerate}
		\item South\hfill\textbf{180}
		\item East\hfill\textbf{90}
		\item West\hfill\textbf{270}
	\end{enumerate}
\end{enumerate}

\section{Time}
Aviation uses the 24-hour clock, where midnight is expressed 0000. The four digit number represents hours and minutes. 
So 5:30\,pm is expressed as 1730 and 6:00\,am is 0600.

\begin{enumerate}
	\item In the 4-figure time group, express:
	\begin{enumerate}
		\item 4.15\,am\hfill\textbf{0415}
		\item 4.45\,pm\hfill\textbf{1645}
		\item 9.30\,pm\hfill\textbf{2130}
		\item 11.40\,pm\hfill\textbf{2340}
	\end{enumerate}
	\item Next we can add the date to this group to make it a 6-figure time-group. 
	An example is 031530. This is 1530 (3.30\,pm) on the 3rd of the month. This time 
	group is regularly used in aviation forecasting and aircraft arrival and departure
	times.
	\begin{enumerate}
		\item Express the date/time right now in a 6-figure time group. \hfill\textbf{272130}
		\item Express 5.45\,pm next Friday.\hfill\textbf{291745}
	\end{enumerate}
	\item Finally, an 8-figure time group incorporates the month. So midday on Christmas Day is 
	expressed as 12251200.
	\begin{enumerate}
		\item What was 3\,pm yesterday?     \hfill\textbf{201902261500}
		\item What is 8.15\,am next Tuesday?\hfill\textbf{201903050815}
	\end{enumerate}
\end{enumerate}

\end{document}


